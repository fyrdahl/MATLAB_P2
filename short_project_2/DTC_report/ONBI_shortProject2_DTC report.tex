\documentclass[3p, twocolumn]{elsarticle}

\usepackage{lineno,hyperref}
\usepackage{amsmath}

\usepackage{pgfplots} 
% and optionally (as of Pgfplots 1.3): 
\pgfplotsset{compat=newest} 
\pgfplotsset{plot coordinates/math parser=false} 
%\newlength\figureheight 
%\newlength\figurewidth

\modulolinenumbers[5]

% \journal{Journal of \LaTeX\ Templates}

%%%%%%%%%%%%%%%%%%%%%%%
%% Elsevier bibliography styles
%%%%%%%%%%%%%%%%%%%%%%%
%% To change the style, put a % in front of the second line of the current style and
%% remove the % from the second line of the style you would like to use.
%%%%%%%%%%%%%%%%%%%%%%%

%% Numbered
%\bibliographystyle{model1-num-names}

%% Numbered without titles
%\bibliographystyle{model1a-num-names}

%% Harvard
%\bibliographystyle{model2-names.bst}\biboptions{authoryear}

%% Vancouver numbered
%\usepackage{numcompress}\bibliographystyle{model3-num-names}

%% Vancouver name/year
%\usepackage{numcompress}\bibliographystyle{model4-names}\biboptions{authoryear}

%% APA style
%\bibliographystyle{model5-names}\biboptions{authoryear}

%% AMA style
%\usepackage{numcompress}\bibliographystyle{model6-num-names}

%% `Elsevier LaTeX' style
\bibliographystyle{elsarticle-num}
%%%%%%%%%%%%%%%%%%%%%%%

\begin{document}

\begin{frontmatter}

\title{Application of Magnetic Resonance Fingerprinting to measure brain oedema}
%\tnotetext[mytitlenote]{Fully documented templates are available in the elsarticle package on \href{http://www.ctan.org/tex-archive/macros/latex/contrib/elsarticle}{CTAN}.}

%% Group authors per affiliation:
%\author{Elsevier\fnref{myfootnote}}
\author{Jack Allen}
\author{Supervisors: Peter Jezzard \& James Kennedy}
\address{University of Oxford, Oxford, UK.}
\fntext[myfootnote]{Since 1880.}

%% or include affiliations in footnotes:
%\author[mymainaddress,mysecondaryaddress]{Elsevier Inc}
%\ead[url]{www.elsevier.com}

%\author[mysecondaryaddress]{Global Customer Service\corref{mycorrespondingauthor}}
%\cortext[mycorrespondingauthor]{Corresponding author}
%\ead{support@elsevier.com}

%\address[mymainaddress]{1600 John F Kennedy Boulevard, Philadelphia}
%\address[mysecondaryaddress]{360 Park Avenue South, New York}


\begin{abstract}

% 'Graphical abstract' as well?
\end{abstract}

% 'Research highlights' section as well?

\begin{keyword}
Magnetic Resonance Imaging, Fingerprinting, Stroke
\end{keyword}

\end{frontmatter}

%\linenumbers

\section{Introduction}

% Brief Introduction to MRI and what it's used for
Magnetic resonance imaging (MRI) has become an established and widely used tool for producing images for diagnosis and disease monitoring in clinical environments. Structural and functional information can be obtained, depending on the method used. One of the uses of MRI is to produce quantitative parameter measurements, in order to distinguish between different tissue types within a sample and provide image contrast. Two commonly extracted properties are the terms that describe the rate at which the transverse and longitudinal magnetisation components evolve over time. These are known as T2 and T1, respectively. The transverse component decays towards zero, according to Eq. \eqref{eq:transverse_decay}, where $t$ is the time elapsed, $M_xy(t)$ is the transverse magnetisation at $t$ and $T_2$ determines the rate of decay. However, in practice $T_2$ is replaced with $T^{*}_{2}$, because spatial inhomogeneities in the static field play a role in dephasing the excited spins.

Equation \eqref{eq:T2star} describes the influence that the relaxation due to $B_0$ inhomogenties has on the total transverse relaxation, where the effect of the inhomogenities is represented by $T^{'}_{2}$.

\begin{equation} \label{eq:transverse_decay}
M_{xy} (t) = M_{xy}(0)\exp^{-\frac{t}{T_{2}}}
\end{equation}


\begin{equation} \label{eq:T2star}
 \frac{1}{T^{*}_{2}} = \frac{1}{T_{2}} + \frac{1}{T^{'}_2}
\end{equation}

Equation \eqref{eq:longitudinal_recovery} describes the behaviour of the longitudinal magnetisiation, where $ t $ is the elapsed time, $M_{z}(0)$ is the initial longitudinal magnetisation and $M_{z,eq}$ is the equilibrium magnetisation that would occur, given $B_0$, but in the absence of excitation pulses. The constant $T_1$ determines the rate at which the longitudinal magnetisation chnages over time.

\begin{equation} \label{eq:longitudinal_recovery}
M_{z} (t) = M_{z,eq}(1 - e^{-\frac{t}{T_1}} ) + M_{z}(0)e^{-\frac{t}{T_1} } 
\end{equation}

 \cite{haackemagnetic}.


% Limitations of convention multi-parameter estimation
Commonly used approaches for measuring T1 and T2 rely on altering the magnetisation with radiofrequency pulses and measuring the signal at different time points afterwards. For example, spin echo (SE) experiments involve two RF pulses before the acquistion of an image. The pulses allow recovery of the signal that is lost due to the dephasing effect from inhomogeneities in $B_0$. The second pulse causes rephasing (an 'echo') at time TE after the first pulse. Parameter measurements usually requiring a significant amount of time.   %explain why T1 and T2 measurements take a long time* 
%describe a spin echo sequence*
%describe an IR sequence*

% Application to AVIC and stroke patient brain analysis
A large portion of clinical MRI research has been directed towards improving the way that the human brain is imaged.
Within the setting of emergency medicine, fast response and diagnosis can improve the prognosis of a patient. The information gained from imaging in this environment is cruicial for ensuring accurate diagnosis and influence the decision making progress about subsequent treatment. Specifically, patients suffering from strokes could benefit from these improvements. The localised reduction in blood supply that occurs during a stroke can cause the affected region of cells to become damaged or die. Cell damage can be exhibited as extracellular water build up (Oedema) or by a change in the pH value of the affected tissue. In order to identify affect areas, it is important to be able to measure markers of damage accurately and precisly.

% Introduce Magnetic resonance fingerprinting and it's potential advantages
Recently, a new approach known as magnetic resonance fingerprinting (MRF) has been developed and applied, which has been used to simultanously measure multiple parameters, such as T1 and T2, with a reduction in scan time \cite{ma2013magnetic} . During a fingerprinting experiment, a pesudo-random seqeunce is used to manipulate the spins in a sample. This can be achieved by varying factors in the seqence, such the time in between subsequent RF pulses, as well the size of any rotations in the magnetisation that may be induced by the RF pulses.

% Dictionary
Once a series of images have been acquired via the fingerprinting sequence, the pixel-wise signal variation over the experiment is compared to a previously built dictionary of time courses, in order to find the closest match. The dictionary is built by simulating the signal expected from the particular experiment type, for variety of difficult combinations of parameters (such as T1 and T2). 


\section{Materials and Method}


\subsection{Phantom production}
% Production of a phantom with a range of T1 of T2 values
In order to test the performance of the method with a sample containing various properties we created a phantom with a range of T1 and T2 values. %*if I manage to make another phantom, mention here that the values are similar to those expected in the brain and cite it * To

\subsection{Dictionary creation}
% MATLAB simulation of signal
Using MATLAB (The
MathWorks, Natick, MA), a dictionary was composed for %*mention T1 and T2 values used*
In order to account for some variations in the $B_1$ field,  deviations in the flip angle of $\pm$30\% (%flipangle deviation increments?
)were also used in the creation of the dictionary.



% Images with varying TE and TI, in order to extract T2 and T1
\subsection{Image aqcuistion}

 Images were acquired of both phantoms, at the acute vascular imaging centre (AVIC) at the John Radcliffe Hospital in Oxford, using a 3T strength Siemens Verio magnet and a 12 channel Siemens head coil. 
To aqcuire images of each phantom that were later used to extract the T1 and T2 values, inversion recovery (IR) and spin echo (SE) experiments were performed.
To produce the images for fingerprinting, timing lists were using to control a spin echo sequence. The timing lists were pre-written and pesudo-random. The TE, TR, and flip angles for each repetition were varied. For each timing list, 48 images were recorded, via 48 repetitions. 

%*How was the 'random' list chosen?*

\subsection{T1 and T2 extraction}


\subsection{Signal matching}



\section{Results and Discussion}

\section{Conclusions}

\section*{References}

\bibliography{ONBI_shortproject2_refs}

\end{document}